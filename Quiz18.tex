\documentclass[20pt]{extarticle}

\usepackage{amsmath, amssymb, amsthm, amsfonts, mathrsfs}
\usepackage{times, flexisym, mdframed, xcolor}
\usepackage{ulem,multicol}
\usepackage{mathtools}
\usepackage{tikz}
\usepackage{hyperref}
\usepackage{graphicx}
\usepackage{bbm}
\usepackage{fancyhdr}
\usepackage{tikz-cd}
\usepackage[doublespacing]{setspace}
\usepackage{pgfplots}
% \usepackage{sectsty}
\setlength{\parindent}{0pt}
\usetikzlibrary{fit, shapes}
\usetikzlibrary{arrows.meta}
\pgfdeclarelayer{background}
\pgfsetlayers{background,main}
\pgfplotsset{compat=1.17}
% \pgfplotsset{soldot/.style={color=black,only marks,mark=*}} \pgfplotsset{holdot/.style={color=black,fill=white,only marks,mark=*}}

\usepackage[paperwidth=400mm,paperheight=300mm,left=10mm,right=150mm, top=10mm, bottom=10mm]{geometry}
\usepackage{draculatheme}
\usepackage[doublespacing]{setspace}

% \sectionfont{\color{draculaorange}}  % sets colour of sections

%Control layout of itemize, enumerate, description (https://ctan.org/pkg/enumitem?lang=en)
\usepackage[shortlabels]{enumitem}
\pagestyle{empty}
\def\endquestion{
  \[\ \]
  \begin{tikzpicture}

    % \foreach \y in {7, 14,...,60}
    % \draw(0,\y mm) -- (80 mm, \y mm);
    % \draw(0,\y mm) -- (170 mm, \y mm);
      \foreach \y in {5, 10,...,60}
          \foreach \x in {5,10,...,170}
          {
              \fill[draculacomment!75] (\x mm,\y mm) circle (.5 pt);
          }
  \end{tikzpicture}
}
\newcommand{\m}{\scalebox{0.5}[1.0]{$-$}}
\newcommand{\abs}[1]{\left|#1\right|}
\newcommand{\LP}{\left(}
\newcommand{\RP}{\right)}
\newcommand{\LS}{\left\lbrace}
\newcommand{\RS}{\right\rbrace}
\newcommand{\LB}{\left[}
\newcommand{\RB}{\right]}
\newcommand{\MM}{\ \middle|\ }

\newcommand{\N}{\mathbb{N}}
\newcommand{\Z}{\mathbb{Z}}
\newcommand{\Q}{\mathbb{Q}}
\newcommand{\R}{\mathbb{R}}
\setlength{\parindent}{0pt}
\usepackage{parskip}
\begin{document}
% \topmargin=-1cm
% \textwidth=170truemm
% \textheight=230truemm

% \begin{large}MATH:1350:XXXX %use your section number
%  \hskip2.7cm{\bf Quiz 1} \hfill{Fall 2021}
% \end{large}
% \vskip0.7cm

% {\bf Name} :

% \hskip1cm ---------------------------------------

% \vskip0.5cm


% --------------------------------------------------------------
%                         Start here
% --------------------------------------------------------------
% \pagestyle{fancy}
% \fancyhf{}
% \rhead{MATH:1005}
% \lhead{Name:\hskip2in ID:}
\noindent
% \section*{\textbf{\color{draculaorange}Types of Functions}}
% \vskip0.7cm
% \subsection*{\textbf{\color{draculaorange}Piecewise Functions}}
% \vskip0.7cm
% \subsection*{\textbf{\color{draculaorange}Notation}}
% \[f(x)=\begin{cases}
%   \text{function},&\text{condition}\\
%   \vdots&\vdots\\
%   \text{function},&\text{condition}\\
% \end{cases}\]
% \newpage
% \subsection*{\textbf{\color{draculared}Example:}}

% \begin{tikzpicture}
%   \draw[very thin,color=draculacomment] (-0.1,-0.1) grid (17,7);

%   \draw[->] (-0.2,0) -- (17,0) node[right] {$t$};
%   \draw[->] (0,-0.2) -- (0,7) node[above] {$\Delta  v$};

%   \draw[color=draculacyan,domain=0:2]    plot (\x,\x)             node[left] {Stage 1};
%   \draw[color=draculayellow,domain=0:4]    plot (\x+2,.5*\x+2)             node[left] {Stage 2};
%   \draw[color=draculared,domain=0:3]    plot (\x+6,.6*\x+4)             node[right] {Stage 3};
%   \draw[color=draculagreen,domain=0:6]    plot (\x+9,-1*\x+5.8)             node[right] {Falling};
%   % \x r means to convert '\x' from degrees to _r_adians:
% \end{tikzpicture}

% \[f(x)=\begin{cases}
%   x,&[0,2)\\
%   \frac{1}{2}x,&(2,6]\\
%   \frac{3}{5}x,&(6,9]\\
%   -x,&(9,15]\\
% \end{cases}\]
% \subsection*{\textbf{\color{draculared}Example:}}
% \begin{tikzpicture}[scale=3]
%   \begin{axis} [
%     domain=-10:10,
%     xmin=-10, xmax=10,
%     ymin=-10, ymax=10,
%     grid=both,
%     % grid style={line width=.08pt, draw=draculacomment!80},
%     major grid style={line width=.5pt,draw=draculacomment},
%     minor grid style={line width=.08pt,draw=draculacomment!80},
%     axis lines=middle,
%     minor tick num=4,
%     enlargelimits={abs=0.5},
%     axis line style={latex-latex},
%     ticklabel style={font=\tiny},
% ]
%     \addplot [domain=-10:2,color=draculacyan, smooth, thick] { x};
%     \addplot [domain=2:10,color=draculared, smooth, thick] { 2*x};
%     \addplot[color=draculacyan,only marks,mark=*] coordinates{(2,2)};
%     \addplot[color=draculared,fill=draculabg,only marks,mark=*] coordinates{(2,4)};
%   \end{axis}
% \end{tikzpicture}
% % \begin{tikzpicture}
% %   \draw[very thin,color=draculacomment] (-0.1,-0.1) grid (10,10);

% %   \draw[->] (-0.2,0) -- (10,0) node[right] {$\ $};
% %   \draw[->] (0,-0.2) -- (0,10) node[above] {$\ $};
% %   % \addplot[holdot] coordinates{(-2,-2)};
% %   % \addplot[soldot] coordinates{(4,1)};
% %   \draw[color=draculacyan,domain=0:2]    plot (\x,\x)             ;
% %   % \node at (2,2)[circle,fill,color=draculacyan,inner sep=2pt]{};
% %   \draw[color=draculayellow,domain=0:8]    plot (\x+2,.5*\x+5);
% %   %  node[circle,draw,color=draculayellow] (c) at (2,5){};
% %     % \draw[color=draculared,domain=0:3]    plot (\x+6,.6*\x+4)             node[right] {Stage 3};
% %   % \draw[color=draculagreen,domain=0:6]    plot (\x+9,-1*\x+5.8)             node[right] {Falling};
% %   % \x r means to convert '\x' from degrees to _r_adians:
% % \end{tikzpicture}
% \newpage
% \subsection*{\textbf{\color{draculaorange}Absolute Value}}
% \subsection*{\textbf{\color{draculared}Example:}}
% \begin{tikzpicture}[scale=3]
%   \begin{axis} [
%     domain=-10:10,
%     xmin=-10, xmax=10,
%     ymin=-10, ymax=10,
%     grid=both,
%     % grid style={line width=.08pt, draw=draculacomment!80},
%     major grid style={line width=.5pt,draw=draculacomment},
%     minor grid style={line width=.08pt,draw=draculacomment!80},
%     axis lines=middle,
%     minor tick num=4,
%     enlargelimits={abs=0.5},
%     axis line style={latex-latex},
%     ticklabel style={font=\tiny},
% ]
% \addplot [domain=0:10,color=draculared, smooth, thick] { x};
% \addplot [domain=-10:0,color=draculared, smooth, thick] { -x};
% \addplot [domain=-10:10,color=draculacyan, smooth, thin] { x};
%     % \addplot[color=draculacyan,only marks,mark=*] coordinates{(2,2)};
%     % \addplot[color=draculared,fill=draculabg,only marks,mark=*] coordinates{(2,4)};
%   \end{axis}
% \end{tikzpicture}
% \newpage
% \subsection*{\textbf{\color{draculared}Example:}}
% \begin{tikzpicture}[scale=3]
%   \begin{axis} [
%     domain=-13:13,
%     xmin=-13, xmax=13,
%     ymin=-1.5, ymax=1.5,
%     grid=both,
%     % grid style={line width=.08pt, draw=draculacomment!80},
%     major grid style={line width=.5pt,draw=draculacomment},
%     minor grid style={line width=.08pt,draw=draculacomment!80},
%     axis lines=middle,
%     minor tick num=4,
%     enlargelimits={abs=0.5},
%     axis line style={latex-latex},
%     ticklabel style={font=\tiny},
% ]
% \addplot [domain=-13:13,color=draculared,samples=200, smooth, thick] { sin(x r)};
% % \addplot [domain=-10:10,color=draculacyan, smooth, thin] { x};
%     % \addplot[color=draculacyan,only marks,mark=*] coordinates{(2,2)};
%     % \addplot[color=draculared,fill=draculabg,only marks,mark=*] coordinates{(2,4)};
%   \end{axis}
% \end{tikzpicture}
% \newpage
% \begin{tikzpicture}[scale=3]
%   \begin{axis} [
%     domain=-13:13,
%     xmin=-13, xmax=13,
%     ymin=-1.5, ymax=1.5,
%     grid=both,
%     % grid style={line width=.08pt, draw=draculacomment!80},
%     major grid style={line width=.5pt,draw=draculacomment},
%     minor grid style={line width=.08pt,draw=draculacomment!80},
%     axis lines=middle,
%     minor tick num=4,
%     enlargelimits={abs=0.5},
%     axis line style={latex-latex},
%     ticklabel style={font=\tiny},
% ]
% \addplot [domain=-13:13,color=draculared,samples=200, smooth, thick] { sin(x r)};
% \addplot [domain=-13:13,color=draculacyan,samples=800, smooth, thin] { abs(sin(x r))};
% % \addplot [domain=-10:10,color=draculacyan, smooth, thin] { x};
%     % \addplot[color=draculacyan,only marks,mark=*] coordinates{(2,2)};
%     % \addplot[color=draculared,fill=draculabg,only marks,mark=*] coordinates{(2,4)};
%   \end{axis}
% \end{tikzpicture}
% \newpage
% \subsection*{\textbf{\color{draculaorange}Even and Odd}}
% \vskip0.7cm
% \subsection*{\textbf{\color{draculaorange}Definition:}}
% A function \(f(x)\) is called:
% \begin{itemize}[align=left]
%   \item[Even:] {If \(\forall x \in D\) we have \(f(\m x)=f(x)\)}
%   \vskip0.7cm
%   \item[Odd:] {If \(\forall x \in D\) we have \(f(\m x)=\m f(x)\)}
%   \vskip0.7cm
%   \item[Neither:] {\(\ \)}
% \end{itemize}
% \vskip2cm
% \subsection*{\textbf{\color{draculaorange}Symmetries:}}
% \begin{itemize}[align=left]
%   \item[Even:] {\(\ \)}
%   \vskip0.7cm
%   \item[Odd:] {\(\ \)}
%   \vskip0.7cm
%   \item[Neither:] {\(\ \)}
% \end{itemize}
% \newpage
% \begin{tikzpicture}[scale=1.3]
%   \begin{axis} [
%     domain=-10:10,
%     xmin=-10, xmax=10,
%     ymin=-10, ymax=10,
%     grid=both,
%     % grid style={line width=.08pt, draw=draculacomment!80},
%     major grid style={line width=.5pt,draw=draculacomment},
%     minor grid style={line width=.08pt,draw=draculacomment!80},
%     axis lines=middle,
%     minor tick num=4,
%     enlargelimits={abs=0.5},
%     % axis line style={latex-latex},
%     yticklabels={,,},
%     xticklabels={,,}
% ]
% % \addplot [domain=-10:10,color=draculacyan, smooth, thin] { x};
%     % \addplot[color=draculacyan,only marks,mark=*] coordinates{(2,2)};
%     % \addplot[color=draculared,fill=draculabg,only marks,mark=*] coordinates{(2,4)};
%   \end{axis}
% \end{tikzpicture}
% \vskip0.7cm
% \begin{tikzpicture}[scale=1.3]
%   \begin{axis} [
%     domain=-10:10,
%     xmin=-10, xmax=10,
%     ymin=-10, ymax=10,
%     grid=both,
%     % grid style={line width=.08pt, draw=draculacomment!80},
%     major grid style={line width=.5pt,draw=draculacomment},
%     minor grid style={line width=.08pt,draw=draculacomment!80},
%     axis lines=middle,
%     minor tick num=4,
%     enlargelimits={abs=0.5},
%     % axis line style={latex-latex},
%     yticklabels={,,},
%     xticklabels={,,}
% ]
% % \addplot [domain=-10:10,color=draculacyan, smooth, thin] { x};
%     % \addplot[color=draculacyan,only marks,mark=*] coordinates{(2,2)};
%     % \addplot[color=draculared,fill=draculabg,only marks,mark=*] coordinates{(2,4)};
%   \end{axis}
% \end{tikzpicture}
% \vskip0.7cm
% \begin{tikzpicture}[scale=1.3]
%   \begin{axis} [
%     domain=-10:10,
%     xmin=-10, xmax=10,
%     ymin=-10, ymax=10,
%     grid=both,
%     % grid style={line width=.08pt, draw=draculacomment!80},
%     major grid style={line width=.5pt,draw=draculacomment},
%     minor grid style={line width=.08pt,draw=draculacomment!80},
%     axis lines=middle,
%     minor tick num=4,
%     enlargelimits={abs=0.5},
%     % axis line style={latex-latex},
%     yticklabels={,,},
%     xticklabels={,,}
% ]
% % \addplot [domain=-10:10,color=draculacyan, smooth, thin] { x};
%     % \addplot[color=draculacyan,only marks,mark=*] coordinates{(2,2)};
%     % \addplot[color=draculared,fill=draculabg,only marks,mark=*] coordinates{(2,4)};
%   \end{axis}
% \end{tikzpicture}
% \newpage
% \newpage
% \subsection*{\textbf{\color{draculaorange}Increasing Decreasing}}
% \vskip0.7cm
% \subsection*{\textbf{\color{draculaorange}Definition:}}
% A function \(f(x)\) on an inteval \(I\) is called:
% \begin{itemize}[align=left]
%   \item[Increasing:] {\(\forall a,b \in I\) if \(a<b\) then \(f(a)<f(b)\)}
%   \vskip0.7cm
%   \item[Decreasing:] {\(\forall a,b \in I\) if \(a<b\) then \(f(a)>f(b)\)}
% \end{itemize}
% \begin{tikzpicture}[scale=1.5]
%   \begin{axis} [
%     domain=-10:10,
%     xmin=-10, xmax=10,
%     ymin=-10, ymax=10,
%     grid=both,
%     % grid style={line width=.08pt, draw=draculacomment!80},
%     major grid style={line width=.5pt,draw=draculacomment},
%     minor grid style={line width=.08pt,draw=draculacomment!80},
%     axis lines=middle,
%     minor tick num=4,
%     enlargelimits={abs=0.5},
%     % axis line style={latex-latex},
%     yticklabels={,,},
%     xticklabels={,,}
% ]
% % \addplot [domain=-10:10,color=draculacyan, smooth, thin] { x};
%     % \addplot[color=draculacyan,only marks,mark=*] coordinates{(2,2)};
%     % \addplot[color=draculared,fill=draculabg,only marks,mark=*] coordinates{(2,4)};
%   \end{axis}
% \end{tikzpicture}
% \begin{tikzpicture}[scale=1.5]
%   \begin{axis} [
%     domain=-10:10,
%     xmin=-10, xmax=10,
%     ymin=-10, ymax=10,
%     grid=both,
%     % grid style={line width=.08pt, draw=draculacomment!80},
%     major grid style={line width=.5pt,draw=draculacomment},
%     minor grid style={line width=.08pt,draw=draculacomment!80},
%     axis lines=middle,
%     minor tick num=4,
%     enlargelimits={abs=0.5},
%     % axis line style={latex-latex},
%     yticklabels={,,},
%     xticklabels={,,}
% ]
% % \addplot [domain=-10:10,color=draculacyan, smooth, thin] { x};
%     % \addplot[color=draculacyan,only marks,mark=*] coordinates{(2,2)};
%     % \addplot[color=draculared,fill=draculabg,only marks,mark=*] coordinates{(2,4)};
%   \end{axis}
% \end{tikzpicture}
% \vskip0.7cm
% \newpage
% \begin{tikzpicture}[scale=4]
%   \begin{axis} [
%     domain=-10:10,
%     xmin=-10, xmax=10,
%     ymin=-10, ymax=10,
%     grid=both,
%     % grid style={line width=.08pt, draw=draculacomment!80},
%     major grid style={line width=.5pt,draw=draculacomment},
%     minor grid style={line width=.08pt,draw=draculacomment!80},
%     axis lines=middle,
%     minor tick num=4,
%     enlargelimits={abs=0.5},
%     axis line style={latex-latex},
%     ticklabel style={font=\tiny},
%     % yticklabels={,,},
%     % xticklabels={,,}
% ]
%   \addplot [domain=-10:-5,color=draculacyan, smooth, thick] { x};
%   \addplot [domain=-5:-2,color=draculagreen, smooth, thick] { (x+5)^2-5};
%   \addplot [domain=-2:1,color=draculaorange, smooth, thick] { 4};
%   \addplot [domain=1:3,color=draculapink, smooth, thick] { -(x-1)^3+4};
%   \addplot [domain=3:4,color=draculapurple, smooth, thick] { (x-3)-4};
%   \addplot [domain=4:5,color=draculared, smooth, thick] { -(x-4)-3};
%   \addplot [domain=5:10,color=draculayellow, smooth, thick] { (x-5)^2-4};
%     % \addplot[color=draculacyan,only marks,mark=*] coordinates{(2,2)};
%     % \addplot[color=draculared,fill=draculabg,only marks,mark=*] coordinates{(2,4)};
%   \end{axis}
% \end{tikzpicture}
% \newpage

% %   \begin{tikzpicture}[scale=2]
% %     \begin{axis} [
% %     domain=-10:10,
% %     xmin=-10, xmax=10,
% %     ymin=-10, ymax=10,
% %     grid=both,
% %     % grid style={line width=.08pt, draw=draculacomment!80},
% %     major grid style={line width=.5pt,draw=draculacomment},
% %     minor grid style={line width=.08pt,draw=draculacomment!80},
% %     axis lines=middle,
% %     minor tick num=4,
% %     enlargelimits={abs=0.5},
% %     axis line style={latex-latex},
% %     ticklabel style={font=\tiny},
% %     % yticklabels={,,},
% %     % xticklabels={,,}
% % ]
% %       \addplot [domain=-10:10,color=draculacyan, smooth, thick,<->] { 1 };
% %     \end{axis}
% %   \end{tikzpicture}
% \newpage
% \section*{\textbf{\color{draculaorange}"Parent" Functions}}
%   \begin{tikzpicture}[scale=2]
%     \begin{axis} [
%     domain=-10:10,
%     xmin=-10, xmax=10,
%     ymin=-10, ymax=10,
%     grid=both,
%     % grid style={line width=.08pt, draw=draculacomment!80},
%     major grid style={line width=.5pt,draw=draculacomment},
%     minor grid style={line width=.08pt,draw=draculacomment!80},
%     axis lines=middle,
%     minor tick num=4,
%     enlargelimits={abs=0.5},
%     axis line style={latex-latex},
%     ticklabel style={font=\tiny},
%     % yticklabels={,,},
%     % xticklabels={,,}
% ]
%       \addplot [domain=-10:10,color=draculacyan, smooth, thick, <->] { x };
%     \end{axis}
%   \end{tikzpicture}
% \begin{tikzpicture}[scale=2]
%     \begin{axis} [
%     domain=-10:10,
%     xmin=-10, xmax=10,
%     ymin=-10, ymax=10,
%     grid=both,
%     % grid style={line width=.08pt, draw=draculacomment!80},
%     major grid style={line width=.5pt,draw=draculacomment},
%     minor grid style={line width=.08pt,draw=draculacomment!80},
%     axis lines=middle,
%     minor tick num=4,
%     enlargelimits={abs=0.5},
%     axis line style={latex-latex},
%     ticklabel style={font=\tiny},
%     % yticklabels={,,},
%     % xticklabels={,,}
% ]
%           \addplot [domain=0:10,color=draculacyan, smooth, thick, ->] { x };
%         \addplot [domain=-10:0,color=draculacyan, smooth, thick, <-] { -x };
%     \end{axis}
%   \end{tikzpicture}


%   \begin{tikzpicture}[scale=2]
%     \begin{axis} [
%     domain=-10:10,
%     xmin=-10, xmax=10,
%     ymin=-10, ymax=10,
%     grid=both,
%     % grid style={line width=.08pt, draw=draculacomment!80},
%     major grid style={line width=.5pt,draw=draculacomment},
%     minor grid style={line width=.08pt,draw=draculacomment!80},
%     axis lines=middle,
%     minor tick num=4,
%     enlargelimits={abs=0.5},
%     axis line style={latex-latex},
%     ticklabel style={font=\tiny},
%     % yticklabels={,,},
%     % xticklabels={,,}
% ]
%           \addplot [domain=0:10,color=draculacyan, smooth, samples=200, thick, ->] { sqrt(x) };
%     \end{axis}
%   \end{tikzpicture}
% \begin{tikzpicture}[scale=2]
%     \begin{axis} [
%     domain=-10:10,
%     xmin=-10, xmax=10,
%     ymin=-10, ymax=10,
%     grid=both,
%     % grid style={line width=.08pt, draw=draculacomment!80},
%     major grid style={line width=.5pt,draw=draculacomment},
%     minor grid style={line width=.08pt,draw=draculacomment!80},
%     axis lines=middle,
%     minor tick num=4,
%     enlargelimits={abs=0.5},
%     axis line style={latex-latex},
%     ticklabel style={font=\tiny},
%     % yticklabels={,,},
%     % xticklabels={,,}
% ]
%           \addplot [domain=-10:10,color=draculacyan, smooth, thick, <->] { x^2 };
%     \end{axis}
%   \end{tikzpicture}
% \begin{tikzpicture}[scale=2]
%     \begin{axis} [
%     domain=-10:10,
%     xmin=-10, xmax=10,
%     ymin=-10, ymax=10,
%     grid=both,
%     % grid style={line width=.08pt, draw=draculacomment!80},
%     major grid style={line width=.5pt,draw=draculacomment},
%     minor grid style={line width=.08pt,draw=draculacomment!80},
%     axis lines=middle,
%     minor tick num=4,
%     enlargelimits={abs=0.5},
%     axis line style={latex-latex},
%     ticklabel style={font=\tiny},
%     % yticklabels={,,},
%     % xticklabels={,,}
% ]
%           \addplot [domain=-10:10,color=draculacyan, smooth, thick, <->] { x^3 };
%     \end{axis}
%   \end{tikzpicture}
% \begin{tikzpicture}[scale=2]
%     \begin{axis} [
%     domain=-10:10,
%     xmin=-10, xmax=10,
%     ymin=-10, ymax=10,
%     grid=both,
%     % grid style={line width=.08pt, draw=draculacomment!80},
%     major grid style={line width=.5pt,draw=draculacomment},
%     minor grid style={line width=.08pt,draw=draculacomment!80},
%     axis lines=middle,
%     minor tick num=4,
%     enlargelimits={abs=0.5},
%     axis line style={latex-latex},
%     ticklabel style={font=\tiny},
%     % yticklabels={,,},
%     % xticklabels={,,}
% ]
%       \addplot [domain=0.034:10,color=draculacyan, smooth, samples = 200, thick, <->] { 1/x };
%       \addplot [domain=-10:-0.034,color=draculacyan, smooth, samples = 200, thick, <->] { 1/x };
%     \end{axis}
%   \end{tikzpicture}
% \begin{tikzpicture}[scale=2]
%     \begin{axis} [
%     domain=-10:10,
%     xmin=-10, xmax=10,
%     ymin=-10, ymax=10,
%     grid=both,
%     % grid style={line width=.08pt, draw=draculacomment!80},
%     major grid style={line width=.5pt,draw=draculacomment},
%     minor grid style={line width=.08pt,draw=draculacomment!80},
%     axis lines=middle,
%     minor tick num=4,
%     enlargelimits={abs=0.5},
%     axis line style={latex-latex},
%     ticklabel style={font=\tiny},
%     % yticklabels={,,},
%     % xticklabels={,,}
% ]
%       \addplot [domain=0.058:10,color=draculacyan, samples = 200, thick, <->] { 1/(x^2) };
%       \addplot [domain=-10:-.058,color=draculacyan, samples = 200, thick, <->] { 1/(x^2) };
%     \end{axis}
%   \end{tikzpicture}
%   \begin{tikzpicture}[scale=2]
%     \begin{axis} [
%     domain=-5:5,
%     xmin=-5, xmax=5,
%     ymin=-5, ymax=5,
%     grid=both,
%     % grid style={line width=.08pt, draw=draculacomment!80},
%     major grid style={line width=.5pt,draw=draculacomment},
%     minor grid style={line width=.08pt,draw=draculacomment!80},
%     axis lines=middle,
%     minor tick num=4,
%     enlargelimits={abs=0.5},
%     axis line style={latex-latex},
%     ticklabel style={font=\tiny},
%     % yticklabels={,,},
%     % xticklabels={,,}
% ]
%       \addplot [domain=-5:5,color=draculacyan, samples = 50, thick, <->] { exp(x) };
%     \end{axis}
%   \end{tikzpicture}
% \newpage
% \section*{\textbf{\color{draculaorange}Translation of functions}}
% \subsection*{\textbf{\color{draculaorange}Shifts:}}
% \begin{itemize}[align=left]
%   \item[\underline{\hspace{3cm}}:] {\(f(x)\to f(x-c)\)\\
%   \begin{tikzpicture}[scale=1.5]
%     \begin{axis} [
%       domain=-10:10,
%       xmin=-10, xmax=10,
%       ymin=-10, ymax=10,
%       grid=both,
%       % grid style={line width=.08pt, draw=draculacomment!80},
%       major grid style={line width=.5pt,draw=draculacomment},
%       minor grid style={line width=.08pt,draw=draculacomment!80},
%       axis lines=middle,
%       minor tick num=4,
%       enlargelimits={abs=0.5},
%       % axis line style={latex-latex},
%       yticklabels={,,},
%       xticklabels={,,}
%   ]
%   \addplot [domain=-10:10,color=draculacyan, smooth, thick] { x^2};
%   \addplot [domain=-10:10,color=draculared, smooth, thick] { (x-2)^2};
%       % \addplot[color=draculacyan,only marks,mark=*] coordinates{(2,2)};
%       % \addplot[color=draculared,fill=draculabg,only marks,mark=*] coordinates{(2,4)};
%     \end{axis}
%   \end{tikzpicture}}
%   \vskip0.7cm
%   \item[\underline{\hspace{3cm}}:] {\(f(x)\to f(x+c)\)\\
%   \begin{tikzpicture}[scale=1.5]
%     \begin{axis} [
%       domain=-10:10,
%       xmin=-10, xmax=10,
%       ymin=-10, ymax=10,
%       grid=both,
%       % grid style={line width=.08pt, draw=draculacomment!80},
%       major grid style={line width=.5pt,draw=draculacomment},
%       minor grid style={line width=.08pt,draw=draculacomment!80},
%       axis lines=middle,
%       minor tick num=4,
%       enlargelimits={abs=0.5},
%       % axis line style={latex-latex},
%       yticklabels={,,},
%       xticklabels={,,}
%   ]
%   \addplot [domain=-10:10,color=draculacyan, smooth, thick] { x^2};
%   \addplot [domain=-10:10,color=draculared, smooth, thick] { (x+2)^2};
%   % \addplot [domain=-10:10,color=draculacyan, smooth, thick] { x};
%       % \addplot[color=draculacyan,only marks,mark=*] coordinates{(2,2)};
%       % \addplot[color=draculared,fill=draculabg,only marks,mark=*] coordinates{(2,4)};
%     \end{axis}
%   \end{tikzpicture}}
%   \newpage
%   \item[\underline{\hspace{3cm}}:] {\(f(x)\to f(x)-c\)\\
%   \begin{tikzpicture}[scale=1.5]
%     \begin{axis} [
%       domain=-10:10,
%       xmin=-10, xmax=10,
%       ymin=-10, ymax=10,
%       grid=both,
%       % grid style={line width=.08pt, draw=draculacomment!80},
%       major grid style={line width=.5pt,draw=draculacomment},
%       minor grid style={line width=.08pt,draw=draculacomment!80},
%       axis lines=middle,
%       minor tick num=4,
%       enlargelimits={abs=0.5},
%       % axis line style={latex-latex},
%       yticklabels={,,},
%       xticklabels={,,}
%   ]
%   \addplot [domain=-10:10,color=draculacyan, smooth, thick] { x^2};
%   \addplot [domain=-10:10,color=draculared, smooth, thick] { (x)^2-6};
%   % \addplot [domain=-10:10,color=draculacyan, smooth, thick] { x};
%       % \addplot[color=draculacyan,only marks,mark=*] coordinates{(2,2)};
%       % \addplot[color=draculared,fill=draculabg,only marks,mark=*] coordinates{(2,4)};
%     \end{axis}
%   \end{tikzpicture}}
%   \vskip0.7cm
%   \item[\underline{\hspace{3cm}}:] {\(f(x)\to f(x)+c\)\\
%   \begin{tikzpicture}[scale=1.5]
%     \begin{axis} [
%       domain=-10:10,
%       xmin=-10, xmax=10,
%       ymin=-10, ymax=10,
%       grid=both,
%       % grid style={line width=.08pt, draw=draculacomment!80},
%       major grid style={line width=.5pt,draw=draculacomment},
%       minor grid style={line width=.08pt,draw=draculacomment!80},
%       axis lines=middle,
%       minor tick num=4,
%       enlargelimits={abs=0.5},
%       % axis line style={latex-latex},
%       yticklabels={,,},
%       xticklabels={,,}
%   ]
%   \addplot [domain=-10:10,color=draculacyan, smooth, thick] { x^2};
%   \addplot [domain=-10:10,color=draculared, smooth, thick] { (x)^2+6};
%   % \addplot [domain=-10:10,color=draculacyan, smooth, thick] { x};
%       % \addplot[color=draculacyan,only marks,mark=*] coordinates{(2,2)};
%       % \addplot[color=draculared,fill=draculabg,only marks,mark=*] coordinates{(2,4)};
%     \end{axis}
%   \end{tikzpicture}}
% \end{itemize}
% \newpage
% \subsection*{\textbf{\color{draculaorange}Stretching and reflection:}}
% For \(1<c\)
% \begin{itemize}[align=left]
%   \item[\underline{\hspace{3cm}}:] {\(f(x)\to f(cx)\)\\
%   \begin{tikzpicture}[scale=1.5]
%     \begin{axis} [
%       domain=-10:10,
%       xmin=-10, xmax=10,
%       ymin=-10, ymax=10,
%       grid=both,
%       % grid style={line width=.08pt, draw=draculacomment!80},
%       major grid style={line width=.5pt,draw=draculacomment},
%       minor grid style={line width=.08pt,draw=draculacomment!80},
%       axis lines=middle,
%       minor tick num=4,
%       enlargelimits={abs=0.5},
%       % axis line style={latex-latex},
%       yticklabels={,,},
%       xticklabels={,,}
%   ]
%   \addplot [domain=-10:10,color=draculacyan, smooth, thick] { x^2};
%   \addplot [domain=-10:10,color=draculared, smooth, thick] { (3*x)^2};
%   % \addplot [domain=-10:10,color=draculacyan, smooth, thick] { x};
%       % \addplot[color=draculacyan,only marks,mark=*] coordinates{(2,2)};
%       % \addplot[color=draculared,fill=draculabg,only marks,mark=*] coordinates{(2,4)};
%     \end{axis}
%   \end{tikzpicture}}
%   \vskip0.7cm
%   \item[\underline{\hspace{3cm}}:] {\(f(x)\to f(\frac{1}{c}x)\)\\
%   \begin{tikzpicture}[scale=1.5]
%     \begin{axis} [
%       domain=-10:10,
%       xmin=-10, xmax=10,
%       ymin=-10, ymax=10,
%       grid=both,
%       % grid style={line width=.08pt, draw=draculacomment!80},
%       major grid style={line width=.5pt,draw=draculacomment},
%       minor grid style={line width=.08pt,draw=draculacomment!80},
%       axis lines=middle,
%       minor tick num=4,
%       enlargelimits={abs=0.5},
%       % axis line style={latex-latex},
%       yticklabels={,,},
%       xticklabels={,,}
%   ]

%   \addplot [domain=-10:10,color=draculacyan, smooth, thick] { x^2};
%   \addplot [domain=-10:10,color=draculared, smooth, thick] { (x/3)^2};
%   % \addplot [domain=-10:10,color=draculacyan, smooth, thick] { x};
%       % \addplot[color=draculacyan,only marks,mark=*] coordinates{(2,2)};
%       % \addplot[color=draculared,fill=draculabg,only marks,mark=*] coordinates{(2,4)};
%     \end{axis}
%   \end{tikzpicture}}
%   \newpage
%   \item[\underline{\hspace{3cm}}:] {\(f(x)\to cf(x)\)\\
%   \begin{tikzpicture}[scale=1.5]
%     \begin{axis} [
%       domain=-10:10,
%       xmin=-10, xmax=10,
%       ymin=-10, ymax=10,
%       grid=both,
%       % grid style={line width=.08pt, draw=draculacomment!80},
%       major grid style={line width=.5pt,draw=draculacomment},
%       minor grid style={line width=.08pt,draw=draculacomment!80},
%       axis lines=middle,
%       minor tick num=4,
%       enlargelimits={abs=0.5},
%       % axis line style={latex-latex},
%       yticklabels={,,},
%       xticklabels={,,}
%   ]
%   \addplot [domain=-10:10,color=draculacyan, smooth, thick] { x^2};
%   \addplot [domain=-10:10,color=draculared, smooth, thick] { 6*(x)^2};
%   % \addplot [domain=-10:10,color=draculacyan, smooth, thick] { x};
%       % \addplot[color=draculacyan,only marks,mark=*] coordinates{(2,2)};
%       % \addplot[color=draculared,fill=draculabg,only marks,mark=*] coordinates{(2,4)};
%     \end{axis}
%   \end{tikzpicture}}
%   \vskip0.7cm
%   \item[\underline{\hspace{3cm}}:] {\(f(x)\to \frac{1}{c}f(x)\)\\
%   \begin{tikzpicture}[scale=1.5]
%     \begin{axis} [
%       domain=-10:10,
%       xmin=-10, xmax=10,
%       ymin=-10, ymax=10,
%       grid=both,
%       % grid style={line width=.08pt, draw=draculacomment!80},
%       major grid style={line width=.5pt,draw=draculacomment},
%       minor grid style={line width=.08pt,draw=draculacomment!80},
%       axis lines=middle,
%       minor tick num=4,
%       enlargelimits={abs=0.5},
%       % axis line style={latex-latex},
%       yticklabels={,,},
%       xticklabels={,,}
%   ]
%   \addplot [domain=-10:10,color=draculacyan, smooth, thick] { x^2};
%   \addplot [domain=-10:10,color=draculared, smooth, thick] { ((x)^2)/6};
%   % \addplot [domain=-10:10,color=draculacyan, smooth, thick] { x};
%       % \addplot[color=draculacyan,only marks,mark=*] coordinates{(2,2)};
%       % \addplot[color=draculared,fill=draculabg,only marks,mark=*] coordinates{(2,4)};
%     \end{axis}
%   \end{tikzpicture}}
%   \item[\underline{\hspace{3cm}}:] {\(f(x)\to -f(x)\)\\
%   \begin{tikzpicture}[scale=1.5]
%     \begin{axis} [
%       domain=-10:10,
%       xmin=-10, xmax=10,
%       ymin=-10, ymax=10,
%       grid=both,
%       % grid style={line width=.08pt, draw=draculacomment!80},
%       major grid style={line width=.5pt,draw=draculacomment},
%       minor grid style={line width=.08pt,draw=draculacomment!80},
%       axis lines=middle,
%       minor tick num=4,
%       enlargelimits={abs=0.5},
%       % axis line style={latex-latex},
%       yticklabels={,,},
%       xticklabels={,,}
%   ]
%   \addplot [domain=-10:10,color=draculacyan, smooth, thick] { x^2};
%   \addplot [domain=-10:10,color=draculared, smooth, thick] { -(x)^2};
%   % \addplot [domain=-10:10,color=draculacyan, smooth, thick] { x};
%       % \addplot[color=draculacyan,only marks,mark=*] coordinates{(2,2)};
%       % \addplot[color=draculared,fill=draculabg,only marks,mark=*] coordinates{(2,4)};
%     \end{axis}
%   \end{tikzpicture}}
%   \item[\underline{\hspace{3cm}}:] {\(f(x)\to f(-x)\)\\
%   \begin{tikzpicture}[scale=1.5]
%     \begin{axis} [
%       domain=-10:10,
%       xmin=-10, xmax=10,
%       ymin=-10, ymax=10,
%       grid=both,
%       % grid style={line width=.08pt, draw=draculacomment!80},
%       major grid style={line width=.5pt,draw=draculacomment},
%       minor grid style={line width=.08pt,draw=draculacomment!80},
%       axis lines=middle,
%       minor tick num=4,
%       enlargelimits={abs=0.5},
%       % axis line style={latex-latex},
%       yticklabels={,,},
%       xticklabels={,,}
%   ]
%   \addplot [domain=-10:10,color=draculacyan, smooth, thick] { x^3};
%   \addplot [domain=-10:10,color=draculared, smooth, thin] { (-x)^3};
%   % \addplot [domain=-10:10,color=draculacyan, smooth, thick] { x};
%       % \addplot[color=draculacyan,only marks,mark=*] coordinates{(2,2)};
%       % \addplot[color=draculared,fill=draculabg,only marks,mark=*] coordinates{(2,4)};
%     \end{axis}
%   \end{tikzpicture}}
% \end{itemize}

% \newpage

% \section*{\textbf{\color{draculaorange}Function Arithmetic}}
% \begin{itemize}
%   \item[]{\(\LP f+g\RP (x)=f(x)+g(x)\)}
%   \item[]{\(\LP f-g\RP (x)=f(x)-g(x)\)}
%   \item[]{\(\LP f\cdot g\RP (x)=f(x)\cdot g(x)\)}
%   \item[]{\(\LP \frac{f}{g}\RP (x)=\frac{f(x)}{g(x)}\)}
% \end{itemize}
% \subsection*{\textbf{\color{draculared}\textbf{Example:}}}
% If we let \(f(x)=x^2\) and \(g(x)=4x^3\)
% \begin{itemize}[align=left]
%   \item[\(\LP f+g\RP (x)=\)]{\underline{\hspace{5cm}}}
%   \item[\(\LP f-g\RP (x)=\)]{\underline{\hspace{5cm}}}
%   \item[\(\LP f\cdot g\RP (x)=\)]{\underline{\hspace{5cm}}}
%   \item[\(\LP \frac{f}{g}\RP (x)=\)]{\underline{\hspace{5cm}}}
% \end{itemize}
% \newpage
% \subsection*{\textbf{\color{draculaorange}Domain}}
% \subsubsection*{\textbf{\color{draculaorange}Addition, multiplication \(+\ -\ \cdot\)}}
% The domain of $f+g$, $f-g$, and $f\cdot g$ is the intersection of the domains of $f$ and $g$. In other words, $f+g$, $f-g$, and $f\cdot g$ are defined wherever both $f$ and $g$ are defined.
% \vskip6cm
% \subsubsection*{\textbf{\color{draculaorange}Division}}
% For $\frac{f}{g}$, the domain is the intersection of the domains of $f$ and $g$ excluding $x$ where $g(x)$ is 0.\\
% \vskip6cm
% \subsection*{\textbf{\color{draculaorange}Range}}
% \newpage

% \subsection*{\textbf{\color{draculaorange}Function Composition}}
% We let \(f(x)\) and \(g(x)\) be two functions with
% \textit{\color{draculayellow}"compatable"} domain and codomain
% \vskip4cm
% the compostion of \(f(x)\) and \(g(x)\) written \[\LP f\circ g\RP(x)\]
% is defined to be
% \[f\LP g\LP x\RP\RP\]
% \begin{center}
%   \includegraphics[height=12cm]{onion.jpeg}
% \end{center}
% \newpage
% \subsection*{\textbf{\color{draculared}Example:}}
% Let \(f(x)=x^2\) and \(g(x)=\sqrt{x}=\underline{\hspace{2cm}}\)
% \vskip4cm
% \(\LP f\circ g\RP(x)=\)\\
% \vskip8cm
% \(\LP g\circ f\RP(x)=\)\\
% \newpage
% \subsection*{\textbf{\color{draculared}Example:}}
% Let \(f(x)=x^2\) and \(g(x)=x+1\)
% \vskip4cm
% \(\LP f\circ g\RP(x)=\)\\
% \vskip8cm
% \(\LP g\circ f\RP(x)=\)\\
% \newpage

% \subsection*{\textbf{\color{draculaorange}Domain}}
% The domain of $f\circ g$ is all $x$ in the domain of $g$ so that $g(x)$ is in the domain of $f$.
% \vskip8cm


% \subsection*{\textbf{\color{draculacyan}Note:}}
% The easiest way to find the domain is usually to write an expression for $(f\circ g)(x)$ and find its domain without simplifying.




% \newpage

% \subsection*{\textbf{\color{draculared}Example:}}
% Let \(f:\R\to\R\) where \(f:x\mapsto \frac{1}{x}\), and
% \(g:\LP 0,\infty\RP\to \R\) where \(g:x\mapsto x^2\).
% \vskip4cm
% Find the domain of \(\LP f\circ g\RP(x)\):


% \newpage

% \subsection*{\textbf{\color{draculacyan}Note:}}
% This is one of the most important skills you NEED to have for calculus 1.

% \begin{center}
%   Easy problems > hard problems
% \end{center}
% \vskip8cm
% further
% \begin{center}
%   2 Easy problems > a hard problem
% \end{center}


% \subsection*{\textbf{\color{draculaorange}1:1 (Injective)}}
% \vskip 2cm
% \begin{center}
%   \begin{tikzpicture}[scale=2]


%     \filldraw[draculacyan] (-1,0) circle (5pt)            ;
%     \filldraw[draculacyan] (-1,1) circle (5pt)            ;
%     \filldraw[draculacyan] (-1,2) circle (5pt)            ;
%     \filldraw[draculacyan] (-1,3) circle (5pt)            ;
%     \filldraw[draculacyan] (-1,4) circle (5pt)            ;
%     \filldraw[draculacyan] (-1,5) circle (5pt)            ;
%     \filldraw[draculared]  ( 1,0) circle (5pt)            ;
%     \filldraw[draculared]  ( 1,1) circle (5pt)            ;
%     \filldraw[draculared]  ( 1,2) circle (5pt)            ;
%     \filldraw[draculared]  ( 1,3) circle (5pt)            ;
%     \filldraw[draculared]  ( 1,4) circle (5pt)            ;
%     \filldraw[draculared]  ( 1,5) circle (5pt)            ;
%     \filldraw[draculared]  ( 1,6) circle (5pt)            ;
%     \filldraw[draculared]  ( 1,7) circle (5pt)            ;

%     \draw[-to,ultra thick] (-.75,0) --  node[above] {$f$} (.75,0);
%     \draw[-to,ultra thick] (-.75,1) --  node[above] {$f$} (.75,1);
%     \draw[-to,ultra thick] (-.75,2) --  node[above] {$f$} (.75,2);
%     \draw[-to,ultra thick] (-.75,3) --  node[above] {$f$} (.75,3);
%     \draw[-to,ultra thick] (-.75,4) --  node[above] {$f$} (.75,4);
%     \draw[-to,ultra thick] (-.75,5) --  node[above] {$f$} (.75,5);

%   \end{tikzpicture}
% \end{center}

% Formally:\\
% A function \(f\) is said to be injective if for \(a,b\in D\), with
% \(f(a)=f(b)\) then \(a=b\).

% \newpage
% \subsection*{\textbf{\color{draculaorange}Inverse Functions}}
% \vskip 2cm

% \begin{center}
%   \begin{tikzpicture}[scale=2]


%     \filldraw[draculacyan] (-1,0) circle (5pt)            ;
%     \filldraw[draculacyan] (-1,1) circle (5pt)            ;
%     \filldraw[draculacyan] (-1,2) circle (5pt)            ;
%     \filldraw[draculacyan] (-1,3) circle (5pt)            ;
%     \filldraw[draculacyan] (-1,4) circle (5pt)            ;
%     \filldraw[draculacyan] (-1,5) circle (5pt)            ;
%     \filldraw[draculared]  ( 1,0) circle (5pt)            ;
%     \filldraw[draculared]  ( 1,1) circle (5pt)            ;
%     \filldraw[draculared]  ( 1,2) circle (5pt)            ;
%     \filldraw[draculared]  ( 1,3) circle (5pt)            ;
%     \filldraw[draculared]  ( 1,4) circle (5pt)            ;
%     \filldraw[draculared]  ( 1,5) circle (5pt)            ;
%     \filldraw[draculared]  ( 1,6) circle (5pt)            ;
%     \filldraw[draculared]  ( 1,7) circle (5pt)            ;

%     \draw[to-,ultra thick] (-.75,0) --  node[above] {$f^{\m 1}$} (.75,0);
%     \draw[to-,ultra thick] (-.75,1) --  node[above] {$f^{\m 1}$} (.75,1);
%     \draw[to-,ultra thick] (-.75,2) --  node[above] {$f^{\m 1}$} (.75,2);
%     \draw[to-,ultra thick] (-.75,3) --  node[above] {$f^{\m 1}$} (.75,3);
%     \draw[to-,ultra thick] (-.75,4) --  node[above] {$f^{\m 1}$} (.75,4);
%     \draw[to-,ultra thick] (-.75,5) --  node[above] {$f^{\m 1}$} (.75,5);

%   \end{tikzpicture}


% \end{center}
% \newpage
% \section*{\textbf{\color{draculaorange}Identity Function}}
% We the function \(f:D\to D\) with \(x\mapsto x\) the "Identity" function
% \[f(x)=x\]
% \[id_D(x)=x\]
% % \[\mathbbm{1}(x)=x\]
% \section*{\textbf{\color{draculaorange}Inverting a function}}
% Let \(f:D\to R\) be an injective function. Then \(g:R\to D\) is an inverse of
% \(f\) if
% \[\LP f\circ g\RP(r)=r\]
% and
% \[\LP g\circ f\RP(d)=d\]
% we write \(g\) as \(f^{\m 1}\).
% \newpage
% \section*{\textbf{\color{draculaorange}Tests for injectivity}}
% Let $f: A \to B$
% \begin{itemize}
%   \item[I]{Algebraic
%         \begin{enumerate}
%           \item {Proving something is NOT injective, find a counterexample:\\
%                 find $x_1, x_2 \in A$ such that $f\left(x_1\right)=f\left(x_2\right)$ BUT $x_1 \neq x_2$}
%           \item{Proving something is injective: Find a contradiction:
%                 \begin{enumerate}
%                   \item{Assume $3 x_1, x_2 \in A, x_1 \neq x_2$ but $f\left(x_1\right)=f\left(x_2\right)$}
%                   \item{Write $f\left(x_1\right)=f\left(x_0\right)$}
%                   \item{Simplify until you find a contradiction.}
%                 \end{enumerate}
%                 }
%         \end{enumerate}
%         }
%         \newpage
%         \item[II]{Graphical
%         \begin{enumerate}
%           \item{Horizontal line test
%           \begin{enumerate}
%             \item{Graph the function}
%             \item{Run a Horizontal line across the graph.}
%           \end{enumerate}
%           }
%         \end{enumerate}
%         }
% \end{itemize}

% \begin{tikzpicture}[scale=2]
%   \begin{axis} [
%         domain=-10:10,
%         xmin=-10, xmax=10,
%         ymin=-10, ymax=10,
%         grid=both,
%         % grid style={line width=.08pt, draw=draculacomment!80},
%         major grid style={line width=.5pt,draw=draculacomment},
%         minor grid style={line width=.08pt,draw=draculacomment!80},
%         axis lines=middle,
%         minor tick num=4,
%         enlargelimits={abs=0.5},
%         axis line style={latex-latex},
%         ticklabel style={font=\tiny},
%         % yticklabels={,,},
%         % xticklabels={,,}
%     ]
% \addplot [domain=-10:10,color=draculacyan,samples=200, smooth, thick] { x^2};
% % \addplot [domain=-10:10,color=draculared,samples=200, smooth, thick] { x};
% % \addplot [domain=0:10,color=draculacyan, smooth, thin] { x^(1/2)};
%     % \addplot[color=draculacyan,only marks,mark=*] coordinates{(2,2)};
%     % \addplot[color=draculared,fill=draculabg,only marks,mark=*] coordinates{(2,4)};
%   \end{axis}
% \end{tikzpicture}
% \newpage
% \begin{tikzpicture}[scale=2]
%   \begin{axis} [
%         domain=-10:10,
%         xmin=-10, xmax=10,
%         ymin=-10, ymax=10,
%         grid=both,
%         % grid style={line width=.08pt, draw=draculacomment!80},
%         major grid style={line width=.5pt,draw=draculacomment},
%         minor grid style={line width=.08pt,draw=draculacomment!80},
%         axis lines=middle,
%         minor tick num=4,
%         enlargelimits={abs=0.5},
%         axis line style={latex-latex},
%         ticklabel style={font=\tiny},
%         % yticklabels={,,},
%         % xticklabels={,,}
%     ]
% \addplot [domain=-10:10,color=draculacyan,samples=200, smooth, thick] { 1/3*x^3};
% % \addplot [domain=-10:10,color=draculared,samples=200, smooth, thick] { x};
% % \addplot [domain=0:10,color=draculacyan, smooth, thin] { x^(1/2)};
%     % \addplot[color=draculacyan,only marks,mark=*] coordinates{(2,2)};
%     % \addplot[color=draculared,fill=draculabg,only marks,mark=*] coordinates{(2,4)};
%   \end{axis}
% \end{tikzpicture}

% \newpage
% \section*{\textbf{\color{draculaorange}Finding an Inverse}}
% \begin{enumerate}
%   \item{
%     Algebraic:
%     \begin{enumerate}
%       \item{
%       Verify \(f\) is injective.
%       }
%       \item{write \(f(x)\) as \(y\)}
%       \item{exchange \(x\) and \(y\)}
%       \item{solve for x}
%       \item{exchange \(x\) and \(y\)}
%       \item{write \(y\) as \(f^{\m 1}\)}
%     \end{enumerate}
%     \subsection*{\textbf{\color{draculared}Example:}}
%     \(f(x)=e^x\)
%     \subsection*{\textbf{\color{draculared}Example:}}
%     \(f(x)=\frac{x+1}{x-1}\)
%     \newpage
%   }
%   \item{
%     Geometric:
%     \begin{enumerate}
%       \item{Graph \(f\)}
%       \item{
%       Verify \(f\) is injective.
%       }
%       \item{Reflect \(f\) across \(id_D\)}
%     \end{enumerate}

%   }
% \end{enumerate}
% \subsection*{\textbf{\color{draculared}Example:}}

% \begin{tikzpicture}[scale=2]
%   \begin{axis} [
%         domain=-10:10,
%         xmin=-10, xmax=10,
%         ymin=-10, ymax=10,
%         grid=both,
%         % grid style={line width=.08pt, draw=draculacomment!80},
%         major grid style={line width=.5pt,draw=draculacomment},
%         minor grid style={line width=.08pt,draw=draculacomment!80},
%         axis lines=middle,
%         minor tick num=4,
%         enlargelimits={abs=0.5},
%         axis line style={latex-latex},
%         ticklabel style={font=\tiny},
%         % yticklabels={,,},
%         % xticklabels={,,}
%     ]
% \addplot [domain=-10:10,color=draculacyan,samples=200, smooth, thick] { x^2};
% \addplot [domain=-10:10,color=draculared,samples=200, smooth, thin] { x};
% % \addplot [domain=0:10,color=draculacyan, smooth, thin] { x^(1/2)};
%     % \addplot[color=draculacyan,only marks,mark=*] coordinates{(2,2)};
%     % \addplot[color=draculared,fill=draculabg,only marks,mark=*] coordinates{(2,4)};
%   \end{axis}
% \end{tikzpicture}
% \newpage
% \subsection*{\textbf{\color{draculared}Example:}}
% \begin{tikzpicture}[scale=2]
%   \begin{axis} [
%         domain=-10:10,
%         xmin=-10, xmax=10,
%         ymin=-10, ymax=10,
%         grid=both,
%         % grid style={line width=.08pt, draw=draculacomment!80},
%         major grid style={line width=.5pt,draw=draculacomment},
%         minor grid style={line width=.08pt,draw=draculacomment!80},
%         axis lines=middle,
%         minor tick num=4,
%         enlargelimits={abs=0.5},
%         axis line style={latex-latex},
%         ticklabel style={font=\tiny},
%         % yticklabels={,,},
%         % xticklabels={,,}
%     ]
% \addplot [domain=-10:10,color=draculacyan,samples=200, smooth, thick] { x^3};
% \addplot [domain=-10:10,color=draculared,samples=200, smooth, thin] { x};
% % \addplot [domain=0:10,color=draculacyan, smooth, thin] { x^(1/2)};
%     % \addplot[color=draculacyan,only marks,mark=*] coordinates{(2,2)};
%     % \addplot[color=draculared,fill=draculabg,only marks,mark=*] coordinates{(2,4)};
%   \end{axis}
% \end{tikzpicture}

% \section*{\textbf{\color{draculaorange}Motivating Example }}
% \begin{tikzpicture}[scale=4]
%   \begin{axis} [
%         domain=-10:10,
%         xmin=-4, xmax=4,
%         ymin=-10, ymax=10,
%         grid=both,
%         % grid style={line width=.08pt, draw=draculacomment!80},
%         major grid style={line width=.5pt,draw=draculacomment},
%         minor grid style={line width=.08pt,draw=draculacomment!80},
%         axis lines=middle,
%         % minor x tick num=3,
%         % minor y tick num=4,
%         enlargelimits={abs=0.5},
%         axis line style={latex-latex},
%         ticklabel style={font=\tiny},
%         % yticklabels={,,},
%         % xticklabels={,,}
%     ]
% \addplot [domain=-10:10,color=draculacyan,samples=200, smooth, thick] { x^2} node[left] {\(f(x)=x^2\)} ;
% \addplot [domain=-10:10,color=draculared,samples=200, smooth, thick] { 4*x-4}  ;
% % \addplot [domain=-10:10,color=draculared,samples=200, smooth, thin] { x};
% % \addplot [domain=0:10,color=draculacyan, smooth, thin] { x^(1/2)};
%     \addplot[color=draculared,only marks,mark=*] coordinates{(2,4)};
%     % \addplot[color=draculapurple,only marks,mark=*] coordinates{(2,4)};
%     % \addplot[color=draculayellow,only marks,mark=*] coordinates{(1,1)};
%     % \addplot[color=draculagreen,only marks,mark=*] coordinates{(3,9)};
%     % \addplot[color=draculapink,only marks,mark=*] coordinates{(1/2,1/4)};
%     % \addplot[color=draculared,fill=draculabg,only marks,mark=*] coordinates{(2,4)};
%   \end{axis}
% \end{tikzpicture}
% \newpage
% \section*{\textbf{\color{draculaorange}Lines}}

% \subsection*{\textbf{\color{draculaorange}Algebraic}}

% \(\underline{\hspace{3cm}}=\underline{\hspace{3cm}}\cdot\underline{\hspace{3cm}}+\underline{\hspace{3cm}}\)
% \vskip2cm
% \subsection*{\textbf{\color{draculaorange}Geometric}}
% \begin{tikzpicture}[scale=2]
%   \begin{axis} [
%         domain=-10:10,
%         xmin=-10, xmax=10,
%         ymin=-10, ymax=10,
%         grid=both,
%         % grid style={line width=.08pt, draw=draculacomment!80},
%         major grid style={line width=.5pt,draw=draculacomment},
%         minor grid style={line width=.08pt,draw=draculacomment!80},
%         axis lines=middle,
%         minor tick num=4,
%         enlargelimits={abs=0.5},
%         axis line style={latex-latex},
%         ticklabel style={font=\tiny},
%         % yticklabels={,,},
%         % xticklabels={,,}
%     ]
% % \addplot [domain=-10:10,color=draculacyan,samples=200, smooth, thick] { x^3};
% % \addplot [domain=-10:10,color=draculared,samples=200, smooth, thin] { x};
% % \addplot [domain=0:10,color=draculacyan, smooth, thin] { x^(1/2)};
%     \addplot[color=draculacyan,only marks,mark=*] coordinates{(5,1)};
%     \addplot[color=draculacyan,only marks,mark=*] coordinates{(-2,-4)};
%     % \addplot[color=draculared,fill=draculabg,only marks,mark=*] coordinates{(2,4)};
%   \end{axis}
% \end{tikzpicture}
% \newpage
% \section*{\textbf{\color{draculaorange}Secant Lines}}
% A line between two points on the graph of a function.\\

% \noindent
% \begin{tikzpicture}[scale=3.5]
%   \begin{axis} [
%         domain=-10:10,
%         xmin=-5, xmax=5,
%         ymin=-10, ymax=10,
%         grid=both,
%         % grid style={line width=.08pt, draw=draculacomment!80},
%         major grid style={line width=.5pt,draw=draculacomment},
%         minor grid style={line width=.08pt,draw=draculacomment!80},
%         axis lines=middle,
%         minor x tick num=1,
%         minor y tick num=4,
%         enlargelimits={abs=0.5},
%         axis line style={latex-latex},
%         ticklabel style={font=\tiny},
%         % yticklabels={,,},
%         % xticklabels={,,}
%     ]
% \addplot [domain=-10:10,color=draculacyan,samples=200, smooth, thick] { x^3-4*x};
% % \addplot [domain=-10:10,color=draculared,samples=200, smooth, thin] { x};
% % \addplot [domain=0:10,color=draculacyan, smooth, thin] { x^(1/2)};
%     \addplot[color=draculared,only marks,mark=*] coordinates{(-2,0)};
%     \addplot[color=draculapurple,only marks,mark=*] coordinates{(2,0)};
%     \addplot[color=draculapurple,only marks,mark=*] coordinates{(-2/3^.5,16*3^.5*9^-1)};
%     \addplot[color=draculapurple,only marks,mark=*] coordinates{(2/3^.5,-16*3^.5*9^-1)};
%     % \addplot[color=draculared,fill=draculabg,only marks,mark=*] coordinates{(2,4)};
%   \end{axis}
% \end{tikzpicture}
% \newpage
% \subsection*{\textbf{\color{draculared}Example:}}
% \noindent
% \begin{tikzpicture}[scale=4]
%   \begin{axis} [
%         domain=-10:10,
%         xmin=-4, xmax=4,
%         ymin=-10, ymax=10,
%         grid=both,
%         % grid style={line width=.08pt, draw=draculacomment!80},
%         major grid style={line width=.5pt,draw=draculacomment},
%         minor grid style={line width=.08pt,draw=draculacomment!80},
%         axis lines=middle,
%         minor x tick num=3,
%         minor y tick num=4,
%         enlargelimits={abs=0.5},
%         axis line style={latex-latex},
%         ticklabel style={font=\tiny},
%         % yticklabels={,,},
%         % xticklabels={,,}
%     ]
% \addplot [domain=-10:10,color=draculacyan,samples=200, smooth, thick] { x^2} node[left] {\(f(x)=x^2\)} ;
% % \addplot [domain=-10:10,color=draculared,samples=200, smooth, thin] { x};
% % \addplot [domain=0:10,color=draculacyan, smooth, thin] { x^(1/2)};
%     \addplot[color=draculared,only marks,mark=*] coordinates{(0,0)};
%     \addplot[color=draculapurple,only marks,mark=*] coordinates{(2,4)};
%     \addplot[color=draculayellow,only marks,mark=*] coordinates{(1,1)};
%     \addplot[color=draculagreen,only marks,mark=*] coordinates{(3,9)};
%     \addplot[color=draculapink,only marks,mark=*] coordinates{(1/2,1/4)};
%     % \addplot[color=draculared,fill=draculabg,only marks,mark=*] coordinates{(2,4)};
%   \end{axis}
% \end{tikzpicture}

% \subsection*{\textbf{\color{draculaorange}Pattern}}

% Do the slopes have any pattern?

% \vskip3cm

% \subsubsection*{\textbf{\color{draculaorange}Algebra}}

% How do we write the rule as an ordered pair?
% \[\LP\ \underline{\hspace{3cm}}\ ,\ \underline{\hspace{3cm}}\ \RP \]
% \[\LP\ \underline{\hspace{3cm}}\ ,\ \underline{\hspace{3cm}}\ \RP \]

% If we write:
% \[P=\LP\ 0\ ,\ 0\ \RP \]
% \[Q=\LP\ \underline{\hspace{3cm}}\ ,\ \underline{\hspace{3cm}}\ \RP \]
% What is the slope for the line between \(P\) and \(Q\)? \(\underline{\hspace{3cm}}\)
% What have we done geometrically?
% \newpage

% \begin{tikzpicture}[scale=4]
%   \begin{axis} [
%         domain=-10:10,
%         xmin=-4, xmax=4,
%         ymin=-10, ymax=10,
%         grid=both,
%         % grid style={line width=.08pt, draw=draculacomment!80},
%         major grid style={line width=.5pt,draw=draculacomment},
%         minor grid style={line width=.08pt,draw=draculacomment!80},
%         axis lines=middle,
%         minor x tick num=3,
%         minor y tick num=4,
%         enlargelimits={abs=0.5},
%         axis line style={latex-latex},
%         ticklabel style={font=\tiny},
%         % yticklabels={,,},
%         % xticklabels={,,}
%     ]
% \addplot [domain=-10:10,color=draculacyan,samples=200, smooth, thick] { x^2} node[left] {\(f(x)=x^2\)} ;
% \addplot [domain=-10:10,color=draculagreen,samples=200, smooth, thin] { 3*x} node[left] {\(f(x)=3x\)} ;
% \addplot [domain=-10:10,color=draculapurple,samples=200, smooth, thin] { 2*x} node[left] {\(f(x)=3x\)} ;
% \addplot [domain=-10:10,color=draculayellow,samples=200, smooth, thin] { x} node[left] {\(f(x)=3x\)} ;
% \addplot [domain=-10:10,color=draculapink,samples=200, smooth, thin] { (1/2)*x} node[left] {\(f(x)=3x\)} ;
% % \addplot [domain=-10:10,color=draculared,samples=200, smooth, thin] { x};
% % \addplot [domain=0:10,color=draculacyan, smooth, thin] { x^(1/2)};
%     \addplot[color=draculared,only marks,mark=*] coordinates{(0,0)};
%     \addplot[color=draculagreen,only marks,mark=*] coordinates{(3,9)};
%     \addplot[color=draculapurple,only marks,mark=*] coordinates{(2,4)};
%     \addplot[color=draculayellow,only marks,mark=*] coordinates{(1,1)};
%     \addplot[color=draculapink,only marks,mark=*] coordinates{(1/2,1/4)};
%     % \addplot[color=draculared,fill=draculabg,only marks,mark=*] coordinates{(2,4)};
%   \end{axis}
% \end{tikzpicture}
% \newpage
% \section*{\textbf{\color{draculaorange}Tangent Lines}}

% The tangent line of a point \(P\) on a curve is a line which touches the curve
% exactly once (locally), and is the "limit"of series of secant
% lines with fixed "base point".

% \begin{tikzpicture}[scale=3]
%   \begin{axis} [
%         domain=-10:10,
%         xmin=-4, xmax=4,
%         ymin=-10, ymax=10,
%         grid=both,
%         % grid style={line width=.08pt, draw=draculacomment!80},
%         major grid style={line width=.5pt,draw=draculacomment},
%         minor grid style={line width=.08pt,draw=draculacomment!80},
%         axis lines=middle,
%         % minor x tick num=3,
%         % minor y tick num=4,
%         enlargelimits={abs=0.5},
%         axis line style={latex-latex},
%         ticklabel style={font=\tiny},
%         % yticklabels={,,},
%         % xticklabels={,,}
%     ]
% \addplot [domain=-10:10,color=draculacyan,samples=200, smooth, thick] { x^2} node[left] {\(f(x)=x^2\)} ;
% % \addplot [domain=-10:10,color=draculared,samples=200, smooth, thin] { 4*x-4}  ;
% % \addplot [domain=-10:10,color=draculared,samples=200, smooth, thin] { x};
% % \addplot [domain=0:10,color=draculacyan, smooth, thin] { x^(1/2)};
%     \addplot[color=draculared,only marks,mark=*] coordinates{(2,4)};
%     % \addplot[color=draculapurple,only marks,mark=*] coordinates{(2,4)};
%     % \addplot[color=draculayellow,only marks,mark=*] coordinates{(1,1)};
%     % \addplot[color=draculagreen,only marks,mark=*] coordinates{(3,9)};
%     % \addplot[color=draculapink,only marks,mark=*] coordinates{(1/2,1/4)};
%     % \addplot[color=draculared,fill=draculabg,only marks,mark=*] coordinates{(2,4)};
%   \end{axis}
% \end{tikzpicture}

% % \newpage
% symbolically
% \[\lim_{Q\to P}m_{QP}=m_P\]
% \subsection*{\textbf{\color{draculared}Example:}}

% for the case \(f(x)=x^2\)  and \(P=\LP\ 0,\ 0\RP\)
% \[m_p=\lim _{x \rightarrow 0} \frac{x^2-0}{x-0}=\lim _{x \rightarrow 0} x=0\]
% \subsection*{\textbf{\color{draculared}Example:}}
% What's the tangent line of \(f(x)=x^2-1\) at \(P=\LP -1,2\RP\)
% \newpage
% \subsection*{\textbf{\color{draculared}Example:}}

% Let \(f(x)=x^3-1\)
% \begin{enumerate}
% \item{  \begin{enumerate}
%     \item{Graph \(f\) by hand.}
%     \item{Let \(P=(1,0)\) eyeball \(m_P\)}
%     \item{Estimate \(m_P\) algebraically.}
%     \item{Write down the tangent line to \(P\).}
%   \end{enumerate}}
%   \item {Is there a \(Q\) where \(m_Q<0\)}
% \end{enumerate}
% \newpage

% \subsection*{\textbf{\color{draculayellow}ICON Questions:}}
% Discussion Question for ICON will be:
% let $y=2 x+3$. Find tangent line at $P=(1,5)$.
% What do you notice? Can you postulate a general rule for tangent lines of
% linear functions?

% \section*{\textbf{\color{draculaorange}Limits}}

% \subsection*{\textbf{\color{draculaorange}Intuition of a limit}}

% Getting "close" and motivation

% \newpage
% \(\)
% \newpage
% \(\)
% \newpage

% \subsection*{\textbf{\color{draculaorange} "Definition"}}
% Let \(f:A\to \R\), where \(A\subseteq \R\). Let \(a\in A\)
% we write
% \[\lim_{x\to a^-} f(x)=L_-\]

% \[\lim_{x\to a^+} f(x)=L_+\]

% If \(L_+=L_-\) we can write
% \[\lim_{x\to a} f(x)=L\]

% \newpage
% \subsection*{\textbf{\color{draculared} Example:}}
% Find:\(\lim_{x\to a^-} f(x)=L_-\quad \lim_{x\to a^+} f(x)=L_+\quad\lim_{x\to a} f(x)=L\)\\
% \begin{tikzpicture}[scale=4]
%   \begin{axis} [
%         domain=-10:10,
%         xmin=-5, xmax=5,
%         ymin=-10, ymax=10,
%         grid=both,
%         % grid style={line width=.08pt, draw=draculacomment!80},
%         major grid style={line width=.5pt,draw=draculacomment},
%         minor grid style={line width=.08pt,draw=draculacomment!80},
%         axis lines=middle,
%         minor tick num=4,
%         enlargelimits={abs=0.5},
%         axis line style={latex-latex},
%         ticklabel style={font=\tiny},
%         % yticklabels={,,},
%         % xticklabels={,,}
%     ]
% \addplot [domain=-10:10,color=draculacyan,samples=200, smooth, thick] { x^3-5*x} node[left] {\(f(x)=x^3-5x\)} ;;
% % \addplot [domain=-10:10,color=draculared,samples=200, smooth, thin] { x};
% % \addplot [domain=0:10,color=draculacyan, smooth, thin] { x^(1/2)};
%     % \addplot[color=draculacyan,only marks,mark=*] coordinates{(5,1)};
%     \addplot[color=draculared,only marks,mark=*] coordinates{(-2,2)};
%     % \addplot[color=draculared,fill=draculabg,only marks,mark=*] coordinates{(2,4)};
%   \end{axis}
% \end{tikzpicture}
% \newpage
% \noindent
% \begin{tikzpicture}[scale=4]
%   \begin{axis} [
%         domain=-10:10,
%         xmin=-3, xmax=-1,
%         ymin=1, ymax=3,
%         grid=both,
%         % grid style={line width=.08pt, draw=draculacomment!80},
%         major grid style={line width=.5pt,draw=draculacomment},
%         minor grid style={line width=.08pt,draw=draculacomment!80},
%         axis lines=middle,
%         minor tick num=4,
%         enlargelimits={abs=0.5},
%         axis line style={latex-latex},
%         ticklabel style={font=\tiny},
%         % yticklabels={,,},
%         % xticklabels={,,}
%     ]
% \addplot [domain=-3:-1,color=draculacyan,samples=200, smooth, thick] { x^3-5*x} node[left] {\(f(x)=x^3-5x\)} ;;
% % \addplot [domain=-10:10,color=draculared,samples=200, smooth, thin] { x};
% % \addplot [domain=0:10,color=draculacyan, smooth, thin] { x^(1/2)};
%     % \addplot[color=draculacyan,only marks,mark=*] coordinates{(5,1)};
%     \addplot[color=draculared,only marks,mark=*] coordinates{(-2,2)};
%     % \addplot[color=draculared,fill=draculabg,only marks,mark=*] coordinates{(2,4)};
%   \end{axis}
% \end{tikzpicture}
% \newpage
% \noindent
% \begin{tikzpicture}[scale=4]
%   \begin{axis} [
%         domain=-10:10,
%         xmin=-2.25, xmax=-1.75,
%         ymin=1.75, ymax=2.25,
%         grid=both,
%         % grid style={line width=.08pt, draw=draculacomment!80},
%         major grid style={line width=.5pt,draw=draculacomment},
%         minor grid style={line width=.08pt,draw=draculacomment!80},
%         axis lines=middle,
%         minor tick num=4,
%         enlargelimits={abs=0.5},
%         axis line style={latex-latex},
%         ticklabel style={font=\tiny},
%         % yticklabels={,,},
%         % xticklabels={,,}
%     ]
% \addplot [domain=-2.25:-1.75,color=draculacyan,samples=200, smooth, thick] { x^3-5*x} node[left] {\(f(x)=x^3-5x\)} ;;
% % \addplot [domain=-10:10,color=draculared,samples=200, smooth, thin] { x};
% % \addplot [domain=0:10,color=draculacyan, smooth, thin] { x^(1/2)};
%     % \addplot[color=draculacyan,only marks,mark=*] coordinates{(5,1)};
%     \addplot[color=draculared,only marks,mark=*] coordinates{(-2,2)};
%     % \addplot[color=draculared,fill=draculabg,only marks,mark=*] coordinates{(2,4)};
%   \end{axis}
% \end{tikzpicture}
% \newpage
% \subsection*{\textbf{\color{draculared} Example:}}
% Give me an example of a function/graph where
% \(\lim_{x\to a^-} f(x)\neq\lim_{x\to a^+} f(x)\)\\
% \begin{tikzpicture}[scale=4]
%   \begin{axis} [
%         domain=-10:10,
%         xmin=-5, xmax=5,
%         ymin=-10, ymax=10,
%         grid=both,
%         % grid style={line width=.08pt, draw=draculacomment!80},
%         major grid style={line width=.5pt,draw=draculacomment},
%         minor grid style={line width=.08pt,draw=draculacomment!80},
%         axis lines=middle,
%         minor tick num=4,
%         enlargelimits={abs=0.5},
%         axis line style={latex-latex},
%         ticklabel style={font=\tiny},
%         % yticklabels={,,},
%         % xticklabels={,,}
%     ]
% % \addplot [domain=-10:10,color=draculacyan,samples=200, smooth, thick] { x^3-5*x} node[left] {\(f(x)=x^3-5x\)} ;;
% % \addplot [domain=-10:10,color=draculared,samples=200, smooth, thin] { x};
% % \addplot [domain=0:10,color=draculacyan, smooth, thin] { x^(1/2)};
%     % \addplot[color=draculacyan,only marks,mark=*] coordinates{(5,1)};
%     % \addplot[color=draculared,only marks,mark=*] coordinates{(-2,2)};
%     % \addplot[color=draculared,fill=draculabg,only marks,mark=*] coordinates{(2,4)};
%   \end{axis}
% \end{tikzpicture}
% \subsection*{\textbf{\color{draculared} Example:}}
% Find:\(\lim_{x\to a^-} f(x)=L_-\quad \lim_{x\to a^+} f(x)=L_+\quad\lim_{x\to a} f(x)=L\)\\
% \begin{tikzpicture}[scale=4]
%   \begin{axis} [
%         domain=-10:10,
%         xmin=-5, xmax=5,
%         ymin=-10, ymax=10,
%         grid=both,
%         % grid style={line width=.08pt, draw=draculacomment!80},
%         major grid style={line width=.5pt,draw=draculacomment},
%         minor grid style={line width=.08pt,draw=draculacomment!80},
%         axis lines=middle,
%         minor tick num=4,
%         enlargelimits={abs=0.5},
%         axis line style={latex-latex},
%         ticklabel style={font=\tiny},
%         % yticklabels={,,},
%         % xticklabels={,,}
%     ]
% \addplot [domain=-10:10,color=draculacyan,samples=200, smooth, thick] { x^3-5*x} node[left] {\(f(x)=x^3-5x\)} ;;
% % \addplot [domain=-10:10,color=draculared,samples=200, smooth, thin] { x};
% % \addplot [domain=0:10,color=draculacyan, smooth, thin] { x^(1/2)};
%     % \addplot[color=draculacyan,only marks,mark=*] coordinates{(5,1)};
%     \addplot[color=draculared,only marks,mark=*] coordinates{(-2,7)};
%     \addplot[color=draculared,fill=draculabg,only marks,mark=*] coordinates{(-2,2)};
%   \end{axis}
% \end{tikzpicture}
% \newpage
% \section*{\textbf{\color{draculaorange} Estimateing limits}}
% \subsection*{\textbf{\color{draculaorange} Graphically}}
% \begin{tikzpicture}[scale=4]
%   \begin{axis} [
%         domain=-10:10,
%         xmin=-10, xmax=10,
%         ymin=-.5, ymax=2,
%         grid=both,
%         % grid style={line width=.08pt, draw=draculacomment!80},
%         major grid style={line width=.5pt,draw=draculacomment},
%         minor grid style={line width=.08pt,draw=draculacomment!80},
%         axis lines=middle,
%         minor tick num=4,
%         enlargelimits={abs=0.5},
%         axis line style={latex-latex},
%         ticklabel style={font=\tiny},
%         % yticklabels={,,},
%         % xticklabels={,,}
%     ]
% \addplot [domain=-10:-.025,color=draculacyan,samples=200, smooth, thick] { sin(x r)/x};
% \addplot [domain=.025:10,color=draculacyan,samples=200, smooth, thick] { sin(x r)/x};
% % \addplot [domain=-10:10,color=draculared,samples=200, smooth, thin] { x};
% % \addplot [domain=0:10,color=draculacyan, smooth, thin] { x^(1/2)};
%     % \addplot[color=draculacyan,only marks,mark=*] coordinates{(5,1)};
%     % \addplot[color=draculared,only marks,mark=*] coordinates{(-2,7)};
%     % \addplot[color=draculared,fill=draculabg,only marks,mark=*] coordinates{(-2,2)};
%   \end{axis}
% \end{tikzpicture}
% \newpage
% \subsection*{\textbf{\color{draculaorange} Tabular}}
%   \begin{table*}[h]
%     \begin{tabular}{c | c | c | c | c}
%     $x$ & $\frac{\sin x}{x}$                   &\(\ \)&$x$ & $\frac{\sin x}{x}$                                             \\
%     \(\ \)& \(\ \)& \(\ \)& \(\ \)& \(\ \)\\ \hline
%     \(\ \)& \(\ \)& \(\ \)& \(\ \)& \(\ \)\\
%     $\pi   $         & 0                       &\(\ \)&$\m \pi   $         & 0                                              \\
%     \(\ \)& \(\ \)& \(\ \)& \(\ \)& \(\ \)\\ \hline
%     \(\ \)& \(\ \)& \(\ \)& \(\ \)& \(\ \)\\
%     $\frac{\pi }{ 2}$& $\frac{2}{\pi}$         &\(\ \)&$\m \frac{\pi }{ 2}$& $\frac{2}{\pi}\ $                                \\
%     \(\ \)& \(\ \)& \(\ \)& \(\ \)& \(\ \)\\ \hline
%     \(\ \)& \(\ \)& \(\ \)& \(\ \)& \(\ \)\\
%            $\frac{\pi }{ 4}$& $\frac{2\sqrt{2}}{\pi}$ &\(\ \)&$\m \frac{\pi }{ 4}$& $\frac{2\sqrt{2}}{\pi}\ $ \\
%     \(\ \)& \(\ \)& \(\ \)& \(\ \)& \(\ \)\\ \hline
%     \(\ \)& \(\ \)& \(\ \)& \(\ \)& \(\ \)\\
%     $\frac{\pi }{ 6}$ &$\frac{3}{\pi}$      &\(\ \)&$\m \frac{\pi }{ 6}$& $\frac{3}{\pi}$         \\
%     \(\ \)& \(\ \)& \(\ \)& \(\ \)& \(\ \)\\ \hline
%     \(\ \)& \(\ \)& \(\ \)& \(\ \)& \(\ \)\\
%     $\vdots$ &\(\vdots\)&\(\ \)&$\vdots$& $\vdots\ $ \\
%     \(\ \)& \(\ \)& \(\ \)& \(\ \)& \(\ \)\\ \hline
%     \(\ \)& \(\ \)& \(\ \)& \(\ \)& \(\ \)\\
%     $0$ &\(1\)&\(\ \)&$0$& $1\ $ \\
%     \end{tabular}
%     \end{table*}
% \newpage
% \subsection*{\textbf{\color{draculaorange} Algebraically}}
% Tomorrow
% \newpage
% \section*{\textbf{\color{draculaorange} Divergence}}
% \begin{tikzpicture}[scale=4]

% \begin{axis} [
%   domain=-10:10,
%   xmin=-10, xmax=10,
%   ymin=-10, ymax=10,
%   grid=both,
%   % grid style={line width=.08pt, draw=draculacomment!80},
%   major grid style={line width=.5pt,draw=draculacomment},
%   minor grid style={line width=.08pt,draw=draculacomment!80},
%   axis lines=middle,
%   minor tick num=4,
%   enlargelimits={abs=0.5},
%   axis line style={latex-latex},
%   ticklabel style={font=\tiny},
%   % yticklabels={,,},
%   % xticklabels={,,}
% ]
% \addplot [domain=-10:-.025,color=draculacyan,samples=200, smooth, thick] { -1/x};
% \addplot [domain=.025:10,color=draculacyan,samples=200, smooth, thick] { 1/x};
% % \addplot [domain=-10:10,color=draculared,samples=200, smooth, thin] { x};
% % \addplot [domain=0:10,color=draculacyan, smooth, thin] { x^(1/2)};
% % \addplot[color=draculacyan,only marks,mark=*] coordinates{(5,1)};
% % \addplot[color=draculared,only marks,mark=*] coordinates{(-2,7)};
% % \addplot[color=draculared,fill=draculabg,only marks,mark=*] coordinates{(-2,2)};
% \end{axis}
% \end{tikzpicture}
% \newpage
% \subsection*{\textbf{\color{draculaorange} Definition}}

% Let $\lim _{x \rightarrow a} f(x)$ exist. The limit is said to diverge if
% $\abs{f(x)} $ gets arbitrarily large as $x\to a$. \\
% There are a few cases:
% \begin{enumerate}
%   \item {$\lim _{x \rightarrow a^-}-f(x)=\infty$}
%   \item{$\lim _{x \rightarrow a^{-}} f(x)=-\infty$}
%   \item {$\lim _{x \rightarrow a^+}-f(x)=\infty$}
%   \item{$\lim _{x \rightarrow a^{+}} f(x)=-\infty$}
%   \item {$\lim _{x \rightarrow a}-f(x)=\infty$}
%   \item{$\lim _{x \rightarrow a} f(x)=-\infty$}
% \end{enumerate}
% These are culled vertical asymptotes.
% \subsection*{\textbf{\color{draculared} Question:}}
% What are some examples for each of the above?


\section*{\textbf{\color{draculaorange}Limits}}

\subsection*{\textbf{\color{draculaorange}Algebra}}
\subsection*{\textbf{\color{draculaorange}Limit Laws}}

\begin{enumerate}
  \item{\[\lim _{x \rightarrow a}(f+g)(x)=\lim _{x \rightarrow a}[f(x)+g(x)]=\lim _{x \rightarrow a} f(x)+\lim _{x \rightarrow a} g(x) \]}
  \item{\[\lim _{x \rightarrow a}(f-g)(x)=\lim _{x \rightarrow a}[f(x)-g(x)]=\lim _{x \rightarrow a} f(x)-\lim _{x \rightarrow a} g(x) \]}
  \item{\[\lim _{x \rightarrow a}(fg)(x)=\lim _{x \rightarrow a}[f(x) g(x)]=\lim _{x \rightarrow a} f(x) \cdot \lim _{x \rightarrow a} g(x) \]}
  \newpage\[\ \]
  \item{\[\lim _{x \rightarrow a} \LP\frac{f}{g}\RP(x)=\lim _{x \rightarrow a} \frac{f(x)}{g(x)}=\frac{\lim _{x \rightarrow a} f(x)}{\lim _{x \rightarrow a} g(x)} \text { if } \lim _{x \rightarrow a} g(x) \neq 0\]}
  \item{\[\lim _{x \rightarrow a}c f(x)=c \lim _{x \rightarrow a} f(x) \]}
  \item{\[\lim _{x \rightarrow a} f(y)=f(y)\]}
  \item{\[\lim _{x \rightarrow c} x=a\]}
\end{enumerate}
\newpage
\subsection*{\textbf{\color{draculared}Example:}}
let \(f(x)=\sin\LP x\RP\), \(g(x)=x\), and \(h(x)=\cos (\pi / 4-x)\)
\begin{enumerate}
  \item{\[\lim _{x \rightarrow 0} \LP\frac{f}{g}\RP(x)=?\]}
  \item{\[\lim _{x \rightarrow 0} h(x)=?\]}
  \item{\[\lim _{x \rightarrow 0} \LB\frac{f(x)h(x)}{g(x)}\RB=?\]}
  \item{\[\lim _{x \rightarrow 0} \LB\frac{f(x)}{g(x)}+h(x)\RB=?\]}
  \item{\[\lim _{x \rightarrow 0} \LB\frac{f(x)}{g(x)h(x)}\RB=?\]}
\end{enumerate}
\newpage
\newpage
\subsection*{\textbf{\color{draculared}Example:}}
True/False:
\[\left(\lim _{x \to a} f(x)\right)^n=\lim _{x \to a}\LP f(x)\RP^n\]
Convince me.
\newpage
\subsection*{\textbf{\color{draculared}Example:}}
True/False:
\[\left(\lim _{x \to a} f(x)\right)^{\frac{1}{n}}=\lim _{x \to a}\LP f(x)\RP^{\frac{1}{n}}\]
Convince me.
\newpage
\subsection*{\textbf{\color{draculaorange}Direct Substitution}}
Sometimes we can evaluate a limit by direct substitution, ie: by
simply playing in a for $x$
\subsection*{\textbf{\color{draculared}Example:}}
\(f(x)=x^2+x+4\)
\[\lim_{x \rightarrow 1} f(x)=\]
\newpage
\subsection*{\textbf{\color{draculared}Example:}}
Convince me that a polynomial \(p(x)\) we can always solve
\[\lim_{x \rightarrow a} p(x)\]
by substitution.
\newpage
\subsection*{\textbf{\color{draculaorange}Cancelation}}

What about \(f(x)=x-1\)
\[\lim_{x\to 1} \frac{f(x)}{f(x)}=?\]
\vskip 5cm
When we have a rational expression, we can cancel like terms in the
numerator and denominator.
\newpage
\subsection*{\textbf{\color{draculared}Example:}}
\begin{align*}
   \lim _{x \rightarrow 2} \frac{x^2-x-2}{x^2+x-6}
  & =\lim _{x \rightarrow 2} \frac{(x-2)(x+1)}{(x-2)(x+3)} \\
  & \\
  & =\lim _{x \rightarrow 2} \frac{x+1}{x+3}\\
  & \\
  &=\frac{3}{5}
\end{align*}
\subsection*{\textbf{\color{draculared}Example:}}
\[\lim _{x \rightarrow-2} \frac{x^2+x-2}{x^2+2 x}=?\]
\newpage
\subsection*{\textbf{\color{draculaorange}"Simplification"}}
Change the way the problem looks
\subsubsection*{\textbf{\color{draculaorange}Conjugate}}
Difference of squares
\[\LP \underline{\hspace{3cm}}\RP\LP \underline{\hspace{3cm}}\RP=\underline{\hspace{3cm}}\]
gives us that
\[\frac{1}{a+\sqrt{b}}=\frac{a-\sqrt{b}}{a^2-\LP\sqrt{b}\RP^2}\]
\textbf{\color{draculapurple}NOTE: This is super common in limit problems}
\newpage
\subsection*{\textbf{\color{draculared}Example:}}
\[\lim _{h+0} \frac{\sqrt{h^2+4}-3}{h^2}\]
\newpage
\section*{\textbf{\color{draculaorange}DCT}}
Let $f(x) \leq g(x) \quad \forall x \in \operatorname{Domain}(f)$.
Let $a \in\operatorname{Domain}(f) \cap\operatorname{Domain}(g)$.
If $\lim _{x \rightarrow a} f(x)$ and $\lim _{x \rightarrow a} g(x)$ exist, then
$$\lim _{x \rightarrow a} f(x) \leq \lim _{x \rightarrow a} g(x)$$
Note: We allow $\infty \leq \infty$ and
$-\infty \leq \infty$ but $\infty \notin-\infty$
\newpage
\subsection*{\textbf{\color{draculaorange}Squeeze}}
\begin{itemize}
  \item{Let $f(x), g(x), h(x)$ all have \(a\) in their domain.}
  \item{Suppose the limit $x \rightarrow a$ exists for $f(x), g(x), h(x)$ with
   $$\lim _{x \rightarrow a} f(x)=L_1$$
   $$\lim _{x \to a} h(x)=L_2$$
   }
  \item{If $f(x) \leq g(x) \leq h(x)$ then
  \[L_1\leq \lim _{x \to a} g(x)\leq L_2\]}
  \item{In particular, if $L_1=L_2=L $ then $ \lim _{x \rightarrow a} g(x)=L$}
\end{itemize}

NOTE: This is often used for trig functions.
\end{document}